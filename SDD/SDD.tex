%%%%%%%%%%%%%%%%%%%%%%%%%%%%%%%%%%%%%%%%%%%%%%%%%%%%%%%%%%%%%%%%%%%%%%%%
%Software Design Document
%Created by Blake Eggemeyer 2012 Feb 16 
%
%%%%%%%%%%%%%%%%%%%%%%%%%%%%%%%%%%%%%%%%%%%%%%%%%%%%%%%%%%%%%%%%%%%%%%%%
\documentclass[12pt]{article}
\usepackage[latin1]{inputenc}
\usepackage{amsmath}
\usepackage{amsfonts}
\usepackage{amssymb}
\usepackage{makeidx}
\usepackage{verbatim}
\usepackage{hyperref} % links
\usepackage{graphicx}
\usepackage{float}
\usepackage{lscape}
%\usepackage{titlesec}
\usepackage{rotating} % text in tables

% table colors
\usepackage[table]{xcolor}
\usepackage{color}
\definecolor{grey}{RGB}{230,230,230}

% make paragraphs less important looking
\usepackage{sectsty}
\paragraphfont{\normalfont}
\subparagraphfont{\normalfont}

\usepackage[T1]{fontenc}

\renewcommand{\sfdefault}{phv}
\renewcommand{\rmdefault}{phv}
\renewcommand{\ttdefault}{pcr}

\renewcommand{\c}{\checkmark}

%type face for code and such things
\newcommand{\e}[1] {{\tt #1}}
%used in table
\newcommand{\s}[1] {\begin{sideways}#1\end{sideways}}

\setlength{\parindent}{0pt}

\topmargin 0in
\headheight 0in
\headsep 0in
\textheight 8.5in
\textwidth 6in
\oddsidemargin 0in
\evensidemargin 0in
\headheight 7pt
\headsep 0in
\footskip .7in

\begin{document}
%%%%%%%%%%%%%%%%%%%%%%%%%%%%%%%%%%%

%%%%%%%%%%%%%%%%%%%%%%%%%%%%%%%%%%%
\begin{titlepage}
\begin{flushright} 
% Title
{\LARGE \bfseries Software Design Document}\\[1.2cm]
{\large \bfseries for}\\[1.2cm]
{\huge \bfseries Community Project Tracking}\\[1.2cm]
{\large \bfseries CS 472}\\
\vfill
{\large \bfseries Draft 0.1}\\[2cm]
\emph{Prepared by:} \\
Tim Grediagin\\
Devin Jones\\
Brent Mello\\
Blake Eggemeyer \\ [3cm]
% Bottom of the page
{\large \today}
\\[2cm]
\end{flushright}
\end{titlepage}
%%%%%%%%%%%%%%%%%%%%%%%%%%%%%%%%%%%
\setcounter{tocdepth}{3}
\setcounter{secnumdepth}{5}
\tableofcontents
\newpage
%%%%%%%%%%%%%%%%%%%%%%%%%%%%%%%%%%%

%http://en.wikipedia.org/wiki/Traceability_matrix
%%%%%%%%%%%%%%%%%%%%%%%%%%%%%%%%%%%%%%%%%%%%%%%%%%%%%%%%%%
\section{Introduction}


\subsection{Overview}
This document is intended to be used by the developer of the TODO software.

\subsection{Stakeholders}
The stakeholder in the design is also the client.

\subsection{Definitions}
\setcounter{paragraph}{0}
\setcounter{subsubsection}{0}
\paragraph{Should:} Requirements with this marker are desired, but not crucial, and will be a part of the final deliverable contingent on time and progress.
\paragraph{User:} The person, or persons, who operate or interact directly with the product.
\paragraph{Will:} Requirements with this marker are guaranteed to be in the final delivered product.
\paragraph{Item:} Item refers to a data object. This object is either an appointment or a task.
\paragraph{CSV:} CSV is an acronym for Comma Separated Value. This is a standard and common format of simple tabular data.

\paragraph{GUI:} Acronym for Graphical User Interface. Used to refer to the look and feel the user experiences.
\paragraph{Immediately:} Immediately refers to actions that will begin as soon as the user has given the input for the action to occur.
\paragraph{Client:} Julie Engfer, the Office Manager for Festival of Fairbanks.
\paragraph{Administrator:} A user with special permissions as specified in section 3.1.2 User Management.
%%%%%%%%%%%%%%%%%%%%%%%%%%%%%%
\paragraph{CPT:} The name of this application.
%\paragraph{User:} A person who uses the software directly. All users will have individual login accounts.
%any alternatives to 'worker'
\paragraph{Worker:} The person(s) responsible for the hours worked in a Project. Users are also Workers, but there may be Workers that are not Users, e.g., "boyscouts."
\paragraph{Project:} Seen at top of timesheet, e.g., "Bicycle Path."
\paragraph{Program:} A project which occurs annually, e.g., "Clean Team."
\paragraph{Activity:} The specific type of work a Worker does, e.g., "Ice Chipping."
\paragraph{Location:} The place where an activity is done, e.g., "CORE 1st - 3rd."
%one or the other?
\paragraph{Tool/Equipment:} The implement used to complete an Activity. Corresponds to "Equipment Used" on the original timesheet.
\paragraph{Comment:} A remark a User may optionally provide on the timesheet.
\paragraph{Timesheet:} The name for the web page on which the various data are entered.

\subsection{References}
The {\tt 1998 - IEEE Standard for Information Technology -- Systems Design -- Software Design Descriptions} was referenced to produce this document.

\subsection{Revision tracking}
\begin{tabular}{|l|r|p{4.6in}|}
\hline
0.1 & Feb 16 & Empty document created.\\
\hline
\end{tabular}

%%%%%%%%%%%%%%%%%%%%%%%%%%%%%%%%%%%%%%%%%%%%%%%%%%%%%%%%%%
\section{Design Considerations}

\subsection{Programming Languages}
The project will be completed using Groovy on Grails to produce a java web application. 

\subsection{Database}
The H2 database manager is the default database manager used by Grails and according to information from \href{http://database-management-systems.findthebest.com/compare/16-30/H2-vs-MySQL}{third parties} is sufficient to handle the task necessary.

\subsection{Project Management}
This project will use \e{Git} version control in conjunction with \e{GitHub} to keep track of changes. The repository can be reached at \url{https://github.com/blake6489/Community-Project-Tracking}.

%%%%%%%%%%%%%%%%%%%%%%%%%%%%%%%%%%%%%%%%%%%%%%%%%%%%%%%%%%
\section{Domains}\label{sec:Domains}
<<<<<<< HEAD
\subsection{User}
There will be two types of user accounts: admin and employee. Both types of accounts will have unique usernames and passwords. Employee accounts will have the following permissions: fill in and submit the current time sheet, and view past timesheets submitted by that specific account. Admin accounts will have the following permissions: create/delete admin or employee accounts, view usernames and passwords of all accounts, view and edit all submitted timesheets, create and submit timesheets on behalf of volunteer group, and access to automated report generation.
\subsection{Role}
\subsection{Project}
=======
\setcounter{paragraph}{0}
\subsection{Activity}
\paragraph{constraints}
\paragraph{name} 250 characters, alphanumeric and spaces 
\paragraph{uniqueName} 250 characters, alphanumeric and spaces  set to lower case
\paragraph{setName}

\setcounter{paragraph}{0}
>>>>>>> a5eaa56a2d6685611487c97347af67b28c4445d9
\subsection{Location}
\paragraph{constraints}
\paragraph{name} 250 characters, alphanumeric and spaces 
\paragraph{uniqueName} 250 characters, alphanumeric and spaces  set to lower case
\paragraph{setName}

\setcounter{paragraph}{0}
\subsection{Project}
\paragraph{constraints}
\paragraph{name} 250 characters, alphanumeric and spaces 
\paragraph{hasMany}

\setcounter{paragraph}{0}
\subsection{Role}
\paragraph{constraints}
\paragraph{mapping}
\paragraph{authority}
\paragraph{type}

\setcounter{paragraph}{0}
\subsection{Timesheet}
\paragraph{constraints}
\paragraph{hasMany}
\paragraph{date}
\paragraph{project}
\paragraph{template}
\paragraph{worker}

\setcounter{paragraph}{0}
\subsection{Tool}
\paragraph{constraints}
\paragraph{name} 250 characters, alphanumeric and spaces 
\paragraph{uniqueName} 250 characters, alphanumeric and spaces  set to lower case
\paragraph{setName}

\setcounter{paragraph}{0}
\subsection{User}
\subsection{Worker}
\paragraph{constraints}
\paragraph{name}

\setcounter{paragraph}{0}
\subsection{Work Record}

\section{Controllers}\label{sec:Controllers}

\subsubsection{Activity}
\paragraph{create}
\paragraph{delete}
\paragraph{edit}
\paragraph{index}
\paragraph{list}
\paragraph{save}
\paragraph{show}
\paragraph{update}

\subsubsection{Home}
\paragraph{grails}
\paragraph{home}
\paragraph{index}

\subsubsection{Location}
\paragraph{create}
\paragraph{delete}
\paragraph{edit}
\paragraph{index}
\paragraph{list}
\paragraph{save}
\paragraph{show}
\paragraph{update}

\subsubsection{Login}
\paragraph{ajaxDenied}
\paragraph{ajaxSuccess}
\paragraph{authfail}
\paragraph{denied}
\paragraph{index}
\paragraph{full}
\paragraph{login}

\subsubsection{Logout}
\paragraph{index}

\subsubsection{Project}
\paragraph{create}
\paragraph{delete}
\paragraph{edit}
\paragraph{index}
\paragraph{list}
\paragraph{save}
\paragraph{show}
\paragraph{update}

\subsubsection{Tool}
\paragraph{create}
\paragraph{delete}
\paragraph{edit}
\paragraph{index}
\paragraph{list}
\paragraph{save}
\paragraph{show}
\paragraph{update}

\subsubsection{User}
\paragraph{create}
\paragraph{delete}
\paragraph{edit}
\paragraph{index}
\paragraph{list}
\paragraph{resetPassword}
\paragraph{save}
\paragraph{show}
\paragraph{update}


%%%%%%%%%%%%%%%%%%%%%%%
\section{Views}
\subsection{Layouts}
Layouts are application-wide views not tied to any specific controller. Layouts will be in the "views/layouts" folder.
\subsubsection{application}
This is the application-wide layout. It's responsible for deciding what layout templates to render based on user access levels.
It must also render flash messages to alleviate other views from having to do so.
\subsubsection{\_login}
This template renders when the user hasn't logged in. It'll contain an ajax form allowing the user to login.
Once the user has successfully logged in, the page will refresh.
\subsubsection{\_admin}
This template renders when an admin is logged in. It'll contain links to parts of the application admins can access.
\subsubsection{\_user}
This template renders when a non-admin is logged in. Not sure what it'll show.


\section{Interface}
\begin{figure}[H]
\begin{center}
\includegraphics[width=7in]{"CPTSiteDiagram"}
\caption{Diagram of user  travel through site}
\label{fig:dia}
\end{center}
\end{figure}
%%%%%%%%%%%%%%%%%%%%%%%%%%%%%%%%%%%%%%%%%%%%%%%%%%%%%%%%%%
\section{Test Cases}


%%%%%%%%%%%%%%%%%%%%%%%%%%%%%%%%%%%%%%%%%%%%%%%%%%%%%%%%%%
\section{Traceability matrix}
\begin{center}
\begin{tabular}{|l||*{12}{c|}|*{4}{c|}}
\hline
\multicolumn{1}{|c||}{ }&
\multicolumn{12}{|c||}{Functional} &
\multicolumn{4}{|c|}{Non-functional}\\
\hline
	& \s{} & \\
\hline
\hline
~\ref{sec:Update timing}	&	\c \\
\hline
\end{tabular}
\end{center}

\end{document}

