%%%%%%%%%%%%%%%%%%%%%%%%%%%%%%%%%%%%%%%%%%%%%%%%%%%%%%%%%%%%%%%%%%%%%%%%
%Software Requirements Document
%Created by Blake Eggemeyer 2012 Feb 10 
%
%%%%%%%%%%%%%%%%%%%%%%%%%%%%%%%%%%%%%%%%%%%%%%%%%%%%%%%%%%%%%%%%%%%%%%%%
\documentclass[12pt]{article}
\usepackage[latin1]{inputenc}
\usepackage{amsmath}
\usepackage{amsfonts}
\usepackage{amssymb}
\usepackage{makeidx}
\usepackage{verbatim}
\usepackage{hyperref}
\usepackage{graphicx}
\usepackage{float}
\usepackage{lscape}
%\usepackage{titlesec}
\usepackage[table]{xcolor}
\usepackage{color}
\definecolor{grey}{RGB}{230,230,230}


\usepackage{sectsty}
\paragraphfont{\normalfont\small}
\setlength{\parskip}{-0.0cm}


\usepackage[T1]{fontenc}

\renewcommand{\sfdefault}{phv}
\renewcommand{\rmdefault}{phv}
\renewcommand{\ttdefault}{pcr}



\setlength{\parindent}{0pt}


\topmargin 0in
\headheight 0in
\headsep 0in
\textheight 9.5in
\textwidth 6in
\oddsidemargin 0in
\evensidemargin 0in
\headheight 7pt
\headsep 0in
\footskip .9in

\begin{document}
%%%%%%%%%%%%%%%%%%%%%%%%%%%%%%%%%%%
%%%%%%%%%%%%%%%%%%%%%%%%%%%%%%%%%%%
\begin{titlepage}
\begin{flushright} 
% Title
{\LARGE \bfseries Software Requirements Document}\\[1.2cm]
{\large \bfseries for}\\[1.2cm]
{\huge \bfseries Community Project Tracking}\\[1.2cm]
{\large \bfseries CS 472}\\
\vfill
{\large \bfseries Draft 0.1}\\[2cm]
\emph{Prepared by:} \\
Tim Grediagin\\
Devin Jones\\
Brent Mello\\
Blake Eggemeyer \\ [3cm]
% Bottom of the page
{\large \today}
\\[2cm]
\end{flushright}
\end{titlepage}
%%%%%%%%%%%%%%%%%%%%%%%%%%%%%%%%%%%
\setcounter{tocdepth}{3}
\setcounter{secnumdepth}{4}
\tableofcontents
\newpage
%%%%%%%%%%%%%%%%%%%%%%%%%%%%%%%%%%%
%%%%%%%%%%%%%%%%%%%%%%%%%%%%%%%%%%%
%%%%%%%%%%%%%%%%%%%%%%%%%%%%%%%%%%%
\section{Introduction}

\subsection{Purpose}
This Software Requirements Document for Community Project Tracking is for use by the customer and development team. This document is a formal listing of the functional and non-functional requirements of the Community Project Tracking software.

\subsection{Scope}
The Community Project Tracking software will assist users in the storing and summary of various activity data.

\subsection{Definitions}

\paragraph{GUI:} Acronym for Graphical User Interface. Used to refer to the look and feel the user experiences.
\paragraph{Immediately:} Immediately refers to actions that will begin as soon as the user has given the input for the action to occur.
\paragraph{Should:} Requirements with this marker are desired, but not crucial, and will be a part of the final deliverable contingent on time and progress.
\paragraph{Client:} Julie Engfer, the Office Manager for Festival of Fairbanks.
\paragraph{User:} The person, or persons, who operate or interact directly with the product.
\paragraph{Administrator:} A user with special permissions as specified in section 3.1.2 User Management. Also referred to as Admin.
\paragraph{Will:} Requirements with this marker are guaranteed to be in the final delivered product.
%%%%%%%%%%%%%%%%%%%%%%%%%%%%%%
\paragraph{CPT:} The name of this application.
%\paragraph{User:} A person who uses the software directly. All users will have individual login accounts.
%any alternatives to 'worker'
\paragraph{Worker:} The person(s) responsible for the hours worked in a Project. Users are also Workers, but there may be Workers that are not Users, e.g., "boyscouts."
\paragraph{Project:} Seen at top of timesheet, e.g., "Bicycle Path."
\paragraph{Program:} A project which occurs annually, e.g., "Clean Team."
\paragraph{Activity:} The specific type of work a Worker does, e.g., "Ice Chipping."
\paragraph{Location:} The place where an activity is done, e.g., "CORE 1st - 3rd."
%one or the other?
\paragraph{Tool/Equipment:} The implement used to complete an Activity. Corresponds to "Equipment Used" on the original timesheet.
\paragraph{Comment:} A remark a User may optionally provide on the timesheet.
\paragraph{Timesheet:} The name for the web page on which the various data are entered.


\subsection{References}
Written with the IEEE Recommended Practice for Software Requirements Specifications as a reference and guide. The Tsunami SWR and RPC Donor SWR were referenced to find appropriate wording for some sections.
%\subsection{Overview}

\subsection{Revision tracking:}
%\rowcolors{1}{grey}{white}
\begin{tabular}{|l|r|p{5in}|}
\hline
0.1 & Feb 10 & Document constructed.\\
\hline
\end{tabular}

%%%%%%%%%%%%%%%%%%%%%%%%%%%%%%%%%%%
%%%%%%%%%%%%%%%%%%%%%%%%%%%%%%%%%%%
%%%%%%%%%%%%%%%%%%%%%%%%%%%%%%%%%%%
\section{Overall description}
\subsection{Product functions}

\rowcolors{1}{grey}{white}
\begin{tabular}{l | l p{5.25in}}
\multicolumn{3}{c}{Priority List}\\
1 &\ref{sec:Data Storage}		& Store data regarding operations \\
2 &\ref{sec:Reports}			& Generate summary reports of the data that has been gathered\\
3 &\ref{sec:Users}				& Manage user accounts\\
4 &\ref{sec:Security}			& Prevent unauthorized viewing or modifying of data. \\
5 &\ref{sec:Interface}			& Highly usable interface\\ 
6 &\ref{sec:Backups}			& Prevent loss or corruption of data\\
7 &\ref{sec:Platform}			& Runs on existing platform\\
\end{tabular}

\subsection{User Characteristics}
The CPT is intended to have a narrow user-base with a small number of administrators and a small number of users.

\subsection{Constraints}


\subsection{Assumptions and Dependencies}
\begin{enumerate}
\item \textbf{Language:} The interface for the user is in English.
\end{enumerate}

\subsection{Apportioning of requirements}

%%%%%%%%%%%%%%%%%%%%%%%%%%%%%%%%%%%
%%%%%%%%%%%%%%%%%%%%%%%%%%%%%%%%%%%
%%%%%%%%%%%%%%%%%%%%%%%%%%%%%%%%%%%
\section{Specific requirements}
The requirements are listed and ordered in a priority list so that their order can be changed at a later date without the section numbers needing to be changed and to allow listing of the priorities in one location.\\
\\

\begin{comment}
The first 2 are to be considered absolutely critical. The software cannot be considered useful if those features are absent.

What are the security implications of storing identifiable time sheet info for employees??
\end{comment}

\subsection{Functional requirements}

\subsubsection{User Management}\label{sec:Users}
\paragraph{} The CPT will have two levels of accounts: general Users and Admins. Everything a User can do can be done by an Admin.
\paragraph{} Accounts will require password authentication.
\paragraph{} The CPT will ship with one Admin account.
\paragraph{} Admins will be able to create other Admin and User accounts.
\paragraph{} Admins will be able to reset User passwords.
\paragraph{} Users will be able to change their own password.

%do this?
%\subsubsection{Project Management}\label{sec:Project Management}
%\paragraph{} Admins will be able to create and delete Projects/Programs.
%\paragraph{} Admins will be able to select which Locations will appear on this Project/Program's timesheet.
%\paragraph{} Admins will be able to select which Activities will appear on this Project/Program's timesheet.

\subsubsection{Data Management}\label{sec:Data Management}
\paragraph{} Admins will be able to create, modify, and delete Projects/Programs.
\paragraph{} Admins will be able to create, modify, and delete Locations.
\paragraph{} Admins will be able to create, modify, and delete Activities.
\paragraph{} Admins will be able to create, modify, and delete Tools.
\paragraph{} Admins will be able to create, modify, and delete Workers.

\subsubsection{Timesheets}\label{sec:Timesheets}
\paragraph{} For a Project/Program on a given day, a User will be able to submit the hours worked with each Tool for each Activity for each Location.
\paragraph{} Users will optionally be able to provide a comment for a Project/Program on a given day.
\paragraph{} Admins will be able to submit hours on behalf of a given Worker.
\paragraph{} Users will be able to browse their previously submitted Timesheets.
\paragraph{} Admins will be able to view and edit the Timesheets of any User.
\paragraph{} When viewing timesheets, the weather for the day will be displayed.

\begin{figure}[H]
\begin{center}
\includegraphics[width=6.5in]{"PaperCopy"}
\caption{Example of former timesheet}
\label{fig:timesheet}
\end{center}
\end{figure}
%finding it convenient to use the word "Timesheet" here.  Need to redo the definition if we stick with this.

\subsubsection{Reports}\label{sec:Reports}
%\paragraph{} The CPT will use the daily input data to generate reports as described by the client. %Too abstract?
\paragraph{} For a given Project/Program over a given time period, Admins will be able to see the total hours worked with each Tool for each Activity for each Location.
 
\subsubsection{Data Storage}\label{sec:Data Storage}
\paragraph{} All data entered into the CPT will be saved to disk.

%how are groups handled

\subsubsection{Security}\label{sec:Security}
\paragraph{} The CPT will have preventative measures against the unauthorized viewing or modifying of stored data.

\subsubsection{Backups}\label{sec:Backups}
\paragraph{} The ability to create and store copies of the task data must be considered.

\subsection{Non-functional requirements}
\subsubsection{Interface}\label{sec:Interface}
\paragraph{} The GUI of the CPI will be user friendly for users operating from standard computers, smart phones, and tablet devices.
\paragraph{} The CPI will be able to display the local weather for a given day.

\begin{figure}[H]
\begin{center}
\includegraphics[width=6.5in]{"CPTWorkexamples"}
\caption{Diagram of user workflow}
\label{fig:dia}
\end{center}
\end{figure}



\subsubsection{Platform}\label{sec:Platform}
Several existing platforms are available. One of which is on site at the clients offices running a Windows 2003 server. The others are a few off site servers that are used for a number of different projects.

\subsection{Performance requirements}
The software must store the input data for the duration necessary to create reports on the data.

\subsection{Software system attributes}

\end{document}



