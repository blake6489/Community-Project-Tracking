\documentclass[12pt,a4paper]{article}
\usepackage[latin1]{inputenc}
\usepackage{amsmath}
\usepackage{amsfonts}
\usepackage{amssymb}
\usepackage{makeidx}
\usepackage{verbatim}
\usepackage{hyperref}
\usepackage{graphicx}
\usepackage{float}
\usepackage{lscape}
%\usepackage{titlesec}

\usepackage{sectsty}
\paragraphfont{\normalfont\small}

\usepackage[T1]{fontenc}

\renewcommand{\sfdefault}{phv}
\renewcommand{\rmdefault}{phv}
\renewcommand{\ttdefault}{pcr}

\setlength{\parindent}{0pt}


\topmargin 0in
\headheight 0in
\headsep 0in
\textheight 9.5in
\textwidth 6in
\oddsidemargin 0in
\evensidemargin 0in
\headheight 7pt
\headsep 0in
\footskip .9in

\begin{document}
%%%%%%%%%%%%%%%%%%%%%%%%%%%%%%%%%%%
\begin{titlepage}
\begin{flushright} 
% Title
{\LARGE \bfseries Software Requirements Document}\\[1.2cm]
{\large \bfseries for}\\[1.2cm]
{\huge \bfseries Community Project Tracking}\\[1.2cm]
{\large \bfseries CS 472}\\
\vfill
{\large \bfseries Draft 0.1}\\[2cm]
\emph{Prepared by:} \\
Tim Grediagin\\
Devin Jones\\
Brent Mello\\
Blake Eggemeyer \\ [3cm]
% Bottom of the page
{\large \today}
\end{flushright}
\end{titlepage}
%%%%%%%%%%%%%%%%%%%%%%%%%%%%%%%%%%%
\setcounter{tocdepth}{3}
\setcounter{secnumdepth}{4}
\tableofcontents
\newpage
%%%%%%%%%%%%%%%%%%%%%%%%%%%%%%%%%%%
%%%%%%%%%%%%%%%%%%%%%%%%%%%%%%%%%%%
%%%%%%%%%%%%%%%%%%%%%%%%%%%%%%%%%%%
\subsection{Purpose}


\subsection{Scope}



\subsection{Definitions}
\begin{enumerate}
\item \textbf{GUI:} Acronym for Graphical User Interface. Used to refer to the look and feel the user experiences.
\item \textbf{Immediately:} Immediately refers to actions that will begin as soon as the user has given the input for the action to occur.
\item \textbf{Should:} Requirements with this marker are desired, but not crucial, and will be a part of the final deliverable contingent on time and progress.
\item \textbf{TBD:} Acronym for To Be Determined. This is used in this document to signify that the information necessary for a part of this document is ``To Be Determined''.
\item \textbf{User:} The person, or persons, who operate or interact directly with the product.
\item \textbf{Will:} Requirements with this marker are guaranteed to be in the final delivered product.
\end{enumerate}

\subsection{References}
Written with the IEEE Recommended Practice for Software Requirements Specifications as a reference and guide. The Tsunami SWR and RPC Donor SWR were referenced to find appropriate wording for some sections.
%\subsection{Overview}

\subsection{Revision tracking:}
\begin{tabular}{|l|r|p{5.5in}|}
\hline
0.1 & Feb 10 & Document constructed.\\
\hline
\end{tabular}

%%%%%%%%%%%%%%%%%%%%%%%%%%%%%%%%%%%
%%%%%%%%%%%%%%%%%%%%%%%%%%%%%%%%%%%
%%%%%%%%%%%%%%%%%%%%%%%%%%%%%%%%%%%
\section{Overall description}
\subsection{Product functions}


\begin{enumerate}
\item TBD
\end{enumerate}
\begin{comment}

a long comment

user accounts w/ paas
general account management
admin reset passwords
web,

staff 
location tools hours !

reports!

input data

backup?

weather - low

\end{comment}

\subsection{User Characteristics}


\subsection{Constraints}

\subsection{Assumptions and Dependencies}
\begin{enumerate}
\item \textbf{Time information:} The software relies on time information from the computer it is running on. Inaccurate clock data will be reflected in the software output. 
\item \textbf{Language:} The interface for the user is in English.
\item \textbf{Platform:} The platform has not been specified.
\end{enumerate}

\subsection{Apportioning of requirements}

%%%%%%%%%%%%%%%%%%%%%%%%%%%%%%%%%%%
%%%%%%%%%%%%%%%%%%%%%%%%%%%%%%%%%%%
%%%%%%%%%%%%%%%%%%%%%%%%%%%%%%%%%%%
\section{Specific requirements}

\subsection{Functional requirements}

\subsubsection{Displayed data for tasks}
\paragraph{---:} ---

\subsubsection{Modification of fields}
\paragraph{---:} ---

\subsubsection{Interactions}

\paragraph{---:} ---

\subsection{Non-functional requirements}
\setcounter{paragraph}{0}

\paragraph{---:} ---


\subsection{Performance requirements}

\subsection{Software system attributes}


\end{document}
