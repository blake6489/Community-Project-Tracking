%%%%%%%%%%%%%%%%%%%%%%%%%%%%%%%%%%%%%%%%%%%%%%%%%%%%%%%%%%%%%%%%%%%%%%%%
%Software Requirements Document
%Created by Blake Eggemeyer 2012 Feb 10 
%
%%%%%%%%%%%%%%%%%%%%%%%%%%%%%%%%%%%%%%%%%%%%%%%%%%%%%%%%%%%%%%%%%%%%%%%%
\documentclass[12pt]{article}
\usepackage[latin1]{inputenc}
\usepackage{amsmath}
\usepackage{amsfonts}
\usepackage{amssymb}
\usepackage{makeidx}
\usepackage{verbatim}
\usepackage{hyperref}
\usepackage{graphicx}
\usepackage{float}
\usepackage{lscape}
%\usepackage{titlesec}
\usepackage[table]{xcolor}
\usepackage{color}
\definecolor{grey}{RGB}{230,230,230}


\usepackage{sectsty}
\paragraphfont{\normalfont\small}

\usepackage[T1]{fontenc}

\renewcommand{\sfdefault}{phv}
\renewcommand{\rmdefault}{phv}
\renewcommand{\ttdefault}{pcr}

\setlength{\parindent}{0pt}


\topmargin 0in
\headheight 0in
\headsep 0in
\textheight 9.5in
\textwidth 6in
\oddsidemargin 0in
\evensidemargin 0in
\headheight 7pt
\headsep 0in
\footskip .9in

\begin{document}
%%%%%%%%%%%%%%%%%%%%%%%%%%%%%%%%%%%
\begin{titlepage}
\begin{flushright} 
% Title
{\LARGE \bfseries Software Requirements Document}\\[1.2cm]
{\large \bfseries for}\\[1.2cm]
{\huge \bfseries Community Project Tracking}\\[1.2cm]
{\large \bfseries CS 472}\\
\vfill
{\large \bfseries Draft 0.1}\\[2cm]
\emph{Prepared by:} \\
Tim Grediagin\\
Devin Jones\\
Brent Mello\\
Blake Eggemeyer \\ [3cm]
% Bottom of the page
{\large \today}
\\[2cm]
\end{flushright}
\end{titlepage}
%%%%%%%%%%%%%%%%%%%%%%%%%%%%%%%%%%%
\setcounter{tocdepth}{3}
\setcounter{secnumdepth}{4}
\tableofcontents
\newpage
%%%%%%%%%%%%%%%%%%%%%%%%%%%%%%%%%%%
%%%%%%%%%%%%%%%%%%%%%%%%%%%%%%%%%%%
%%%%%%%%%%%%%%%%%%%%%%%%%%%%%%%%%%%
\section{Introduction}

\subsection{Purpose}
This Software Requirements Document for Community Project Tracking is for use by the customer and development team. This document is a formal listing of the functional and non-functional requirements of the Community Project Tracking system.

\subsection{Scope}
The Community Project Tracking software will assist users in the storing and summary of various activity data.

\subsection{Definitions}
\begin{enumerate}
\item \textbf{GUI:} Acronym for Graphical User Interface. Used to refer to the look and feel the user experiences.
\item \textbf{Immediately:} Immediately refers to actions that will begin as soon as the user has given the input for the action to occur.
\item \textbf{Should:} Requirements with this marker are desired, but not crucial, and will be a part of the final deliverable contingent on time and progress.
\item \textbf{TBD:} Acronym for To Be Determined. This is used in this document to signify that the information necessary for a part of this document is ``To Be Determined''.
\item \textbf{User:} The person, or persons, who operate or interact directly with the product.
\item \textbf{Will:} Requirements with this marker are guaranteed to be in the final delivered product.
\end{enumerate}

\subsection{Definitions}
\begin{enumerate}
\item \textbf{Program:}  
\item \textbf{Project:}  
\item \textbf{Task:}  
\item \textbf{Tool/Equipment:}  
\end{enumerate}

\subsection{References}
Written with the IEEE Recommended Practice for Software Requirements Specifications as a reference and guide. The Tsunami SWR and RPC Donor SWR were referenced to find appropriate wording for some sections.
%\subsection{Overview}

\subsection{Revision tracking:}
%\rowcolors{1}{grey}{white}
\begin{tabular}{|l|r|p{5in}|}
\hline
0.1 & Feb 10 & Document constructed.\\
\hline
\end{tabular}

%%%%%%%%%%%%%%%%%%%%%%%%%%%%%%%%%%%
%%%%%%%%%%%%%%%%%%%%%%%%%%%%%%%%%%%
%%%%%%%%%%%%%%%%%%%%%%%%%%%%%%%%%%%
\section{Overall description}
\subsection{Product functions}


\begin{enumerate}
\item TBD
\end{enumerate}

\subsection{User Characteristics}


\subsection{Constraints}

\subsection{Assumptions and Dependencies}
\begin{enumerate}
\item \textbf{Language:} The interface for the user is in English.
\item \textbf{Platform:} 
\end{enumerate}

\subsection{Apportioning of requirements}

%%%%%%%%%%%%%%%%%%%%%%%%%%%%%%%%%%%
%%%%%%%%%%%%%%%%%%%%%%%%%%%%%%%%%%%
%%%%%%%%%%%%%%%%%%%%%%%%%%%%%%%%%%%
\section{Specific requirements}
The requirements are listed and ordered in a priority list so that their order can be changed at a later date without the section numbers needing to be changed and to allow listing of the priorities in one location.\\
\\
\rowcolors{1}{grey}{white}
\begin{tabular}{l|lp{5.25in}}
\multicolumn{3}{c}{Priority List}\\
1 &\ref{sec:Store Data}			& Store data regarding operations \\
2 &\ref{sec:Reports}			& Generate summary reports of the data that has been gathered\\
3 &\ref{sec:Security}			& Take steps to prevent malicious tampering\\
4 &\ref{sec:Interface}			& Highly usable interface\\ 
5 &\ref{sec:Backups}			& Prevent loss or corruption of data\\
6 &\ref{sec:Platform}			& Runs on existing platform\\
\end{tabular}
\begin{comment}
The first 2 are to be considered absolutely critical. The software cannot be considered useful if those features are absent.

What are the security implications of storing identifiable time sheet info for employees??
\end{comment}

\subsection{Functional requirements}
 
\subsubsection{Store Data}\label{sec:Store Data}
\paragraph{Staff Members:} Data regarding a staff members work day will be collected and stored.

\subsubsection{Reports}\label{sec:Reports}

\subsubsection{Security}\label{sec:Security}
The impact of storing personally identifiable time sheet data must be considered.

\subsubsection{Backups}\label{sec:Backups}

\subsection{Non-functional requirements}
\subsubsection{Interface}\label{sec:Interface}
\paragraph{User friendly:} Determine what User wants from the interface.

\subsubsection{Platform}\label{sec:Platform}
Several existing platforms are available. One of which is on site at the clients offices running a Windows 2003 server. The others are a few off site servers that are used for a number of different projects.

\subsection{Performance requirements}

\subsection{Software system attributes}

\end{document}







\begin{comment}

a long comment

user accounts w/ passwords
general account management
admin reset passwords
web,

staff 
location tools hours !

reports!

input data

backup?

weather - low

\end{comment}
































